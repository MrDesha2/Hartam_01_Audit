
\newpage

\chapter{Описание архитектуры информационных систем}




\section{Описание нормативно-справочной информации (НСИ)}

\subsubsection{Справочник <<Номенклатура>>}

Справочник <<Номенклатура>> ведется в системе 1С:Бухгалтерия.
Планируется заполнять справочник в системе 1С:УНФ.

% Справочник совпадает по готовой продукции только в системе 1С: 7.7 ''Бухгалтерия'' и 1С: 7.7 ''Производство+Бухгалтерия+Услуги'', куда он выгружается автоматически через файл MS Excel из системы 1С: 7.7 ''Бухгалтерия''.

% Во всех системах позиции справочника создается вручную. НСИ не согласованы.

Справочник содержит информацию по:
\begin{itemize}
\item основным материалам %(картон, бумага плоских слоев);
\item вспомогательным материалам (клей, материалы для упаковки и т.д.);
\item услугам;
\item готовой продукции.
\end{itemize}

\subsubsection{Справочник <<Контрагенты>>}

Справочник <<Контрагенты>> ведется в системе 1С:Бухгалтерия.
Планируется заполнять справочник в системе 1С:УНФ.

% В каждой базе ведется свой справочник. НСИ не согласованы.
Справочник содержит информацию по:

\begin{itemize}
\item заказчикам продукции;
\item поставщикам материалов;
\item другим организациям и физическим лицам.
\end{itemize}

\subsection{Технологические карты (Спецификации)}

Вся технологическая информация по изготавливаемым изделиям содержится в электронном виде в формате MS EXCEL.  Файлы находятся в сетевом доступе в разных местах хранения.

Единого места хранения не предусмотрено.
%Форма технологической карты представлена на рис. \ref{pic:TK_1} --- \ref{pic:TK_2}.

%\begin{figure}
%\begin{center}
%\ifnum\pdfoutput=0
%%  \includegraphics[40,0][366,292]{Pics/TK0001.jpg}
%\else 
%%  \includegraphics[height=0.94\textheight, keepaspectratio]{Pics/TK0001.jpg}
%\fi
%\end{center}
%  \caption{Пример спецификации по изготавливаемым изделиям}
%  \label{pic:tk}
%\end{figure}
%\clearpage

Основные параметры по технологическим картам
\begin{enumerate}
\item Носитель --- электронный файл и бумажный вариант;
\item Кодирование ---  неформализуемая система кодификации;
\item Поддержка --- дизайнер, конструктор;
%\item Виды нормируемых ресурсов --- только сырье (картон, бумага для гофрирования);
\item Полнота спецификации относительно переменных затрат --- в спецификации содержатся только марка картона и размеры изделия. Нормировано потребление сырья для марок картона.
\item Вариативность и изменение ТК во времени --- ТК изменяется каждый раз при появлении нового заказа.
\item Несколько ТК на номенклатуру --- нет.
\end{enumerate}



\section{Используемые программные продукты}

\subsection{Программы для организации производства}

Для выполнения следующих функций используются таблицы MS EXCEL:
\begin{enumerate}
\item Месячное планирование продаж;
\item Месячное планирование отгрузки;
\item Формирование портфеля заказов;
\item Суточное планирование производства;
\item Подготовка производства (разработка спецификации изделия);
\item Расчет нормативно-плановой себестоимости продукции;
%\item Расчет нормативно-фактической себестоимости продукции;
\item Суточное планирование работы гофроагрегата;
\item Суточное планирование работы перерабатывающих линий;
\item Складской учет материалов и готовой продукции;
%\item Учет претензий контрагентов;
\item Учет выработки на линиях переработки.
% \item Контроль качества.
%\item Производственный учет выпуска готовой продукции.
\end{enumerate}

%Программная система ПОЯС.
%\begin{enumerate}
%\item Планирование раскроев на гофроагрегат в цехе №4.
%\end{enumerate}

Программа Corel Draw.
\begin{enumerate}
\item Разработка дизайна печати технологических карт.


\end{enumerate}

%Программа ArtiosCAD.
%\begin{enumerate}
%\item Разработка раскроев технологических карт сложной высечки.
%\end{enumerate}

% Программа AutoCAD.
% \begin{enumerate}
% \item Разработка штампов технологических карт сложной высечки.
% \end{enumerate}

\subsection{Учетные системы}

\subsubsection{1С:Бухгалтерия 8.3}


Для выполнения учетных функций производства применяется учетная система на
базе 1С:Бухгалтерия. В учетной системе выполняются следующие функции:

% В систему внесены изменения в виде расширений конфигурации.

\begin{enumerate}
\item Бухгалтерский учет материалов;
\begin{enumerate}
\item Поступление ТМЦ;
\item Перемещение ТМЦ;
\item Списание ТМЦ в производство;
\end{enumerate}
%\item Складской учет полуфабрикатов (переделов);
%\begin{enumerate}
%\item Поступление готовой продукции;
%%\item Списание ТМЦ в производство;
%\item Перемещение ТМЦ;
%\end{enumerate}
\item Складской учет готовой продукции;
\begin{enumerate}
\item Оприходование готовой продукции;
\item Отгрузка готовой продукции;
\end{enumerate}
\item Расчет фактической себестоимости продукции;
\item Управление денежными средствами;
% \item Учет основных средств;
\item Управление продажами;
\item Расчеты с поставщиками, подрядчиками и прочими организациями.
% \item Кадровый учет персонала;
% \item Начисление заработной платы работникам;
%\item Начисление налогов с фонда оплаты труда.
% \item Модуль ''Управление взаимоотношениями с клиентами'' (CRM).
%  \item Производственный учет:
%  \begin{enumerate}
% \item Заявки покупателей;
% \item Заявки на отгрузку;
% \item Раскрои на гофроагрегат;
% \item Заказы на поставку заготовок;
% \item Сменный рапорт на линию;

% \item Кадровый учет персонала;
% \item Начисление заработной платы работникам;
% \item Начисление налогов с фонда оплаты труда.
% \end{enumerate}
\end{enumerate}
%

Автоматизированы следующие рабочие места:
\begin{enumerate}
\item Бухгалтерия;
% \item Отдел учета.
% \item Отдел продаж;
%\item Инженер по подготовке производства;
%\item Инженер по организации управления производством;
%\item Машинист гофроагрегата;
% \item Машинист технологической линии (Ishikawa и Минилайн LMC);
%\item Бухгалтер производства.
% \item Склад сырья;
% \item БППП;
% \item Склад готовой продукции;
%\item Логист.
\end{enumerate}


% \subsubsection{1С: Предприятие 8.3 ''Бухгалтерия предприятия''}

% Бухгалтерский учет ведется удаленно на аутсорсинге в системе 1С: Предприятие 8.3 ''Бухгалтерия предприятия''.
% Выполняются стандартные функции по ведению бухгалтерского учета.

% \begin{enumerate}
% % \item Объемное планирование производства;
% \item Управление продажами;
% \item Управление денежными средствами;
% \item Расчеты с поставщиками, подрядчиками и прочими организациями;
% \item Бухгалтерский учет;
% \item Налоговый учет.
% \end{enumerate}

% Автоматизированы следующие рабочие места:
% \begin{enumerate}
% \item Отдел учета;
% \item Отдел продаж и закупок.
% \end{enumerate}




%
\subsubsection{1С:Зарплата и управление персоналом 8}
В учетной системе выполняются следующие функции.
\begin{enumerate}
\item Кадровый учет персонала;
\item Начисление заработной платы работникам;
\item Начисление налогов с фонда оплаты труда.
\end{enumerate}

Автоматизированы следующие рабочие места:
\begin{enumerate}
\item Бухгалтерия.
\end{enumerate}


\ifx \notincludehead\undefined
\normalsize
\end{document}
\fi 