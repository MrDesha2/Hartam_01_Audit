\newpage

\chapter{Заключение}

\section{Описание <<узких мест>> производства}



\begin{enumerate}

\item Предприятие новое и бизнес-процессы полностью не сформированы.

\item Все данные переносятся вручную, от заведения заказов, до планирования.

\item 	Не ведётся учет неснижаемого запаса по материалам.
%Списание происходит раз в месяц и при заказе вспомогательных материалов закупщику не на что ориентироваться.

\item 	Не ведется журнал выдачи-приемки оснастки.

\item 	Не ведется пробег по оснастке.

\item  Ремонт оснастки не фиксируется.
 
\item  Не ведется учет заявок оснастки.
 
 \item  Нормы и композиции сырья отсутствуют.
 
 \item  Отсутствует расчет объемов необходимой краски на заказ.
 
 %\item Прием краски осуществляет техник УВФ, а списание в бухгалтерии. Есть вероятность появления ошибки при поступлении.

%\item 	ТК хранятся на УВФ в распечатанном виде. При внесении изменений ТК меняется на новую вручную, в связи с чем существует риск выпуска продукции по старой версии ТК.

%\item  Основная ТК и лист укладки на поддон  

%\item  Отсутствуют номера ТК. Необходимую ТК ищут по названию клиента и размерам. Есть вероятность не найти нужную ТК. 

%\item  На сервере ТК в электронном виде хранятся по раздельности (четырехклапанный короб, сложная высечка, дизайн, упаковка). Отсутствует единое хранилище технологической документации.

\item  Бирки ГП печатает мастер смены. Существует риск ошибиться с номенклатурой и количеством.

\item Сырье списывается на производство целыми рулонами, остатки рулонов не взвешиваются, нет весового учета остатков.

\item Не ведется расход сырья на производстве.

\item На производстве не выявлено четкого учета брака.

\item  Расчет необходимого объема сырья (бумага, картон) бригадир ГА осуществляет в уме.

\item  Бирка ПФ - отсутствует количество на поддоне.

\item  Для складской службы не выделены помещения в местах хранения ТМЦ как по готовой продукции, так по сырью (бумага и картон), вспомогательным материалам.

%\item  Отсутствие планирования линий переработки.  На смену могут быть заказы на оду, две линии.  Остальные линии могут простаивать.

%\item  Служба ОТК не участвует в приемки сырья. Измерения проводят по факту дефектов на ГА.


%\item 	В ярлыке на ГП мастер меняет количество на поддоне в зависимости от транспорта под загрузку. Отсутствует четкий транспортный пакет.

%\item 	При физической сверке заготовки в цеху выявляются ошибки, допущенные при списании или оприходовании заготовки, подтягивается не та номенклатура.

%\item 	На производстве не ведется анализ простоев.

%\item На производстве отсутствует позаказный учет производства. 

%\item 	Не правильная организация работы станка ISHIKAWA. Станок вырубает и отдельная группа складывает продукцию, действия не согласованы. При таком количестве персонала можно организовать работу в один поток.

\item 	Макулатура не взвешивается, нет четкого учета, отсутствуют весы. 

%\item 	Списание недостачи заготовок в конце месяца, отсутствие данных по потреблению заготовки в режиме одного заказа.

\item 	Отсутствует план по проведению обслуживания линий (ремонты, смазки).

%\item 	На момент обследования журнал замечаний по работе оборудования не ведется.

\item 	Отсутствует фиксация претензий от потребителей.

%\item ТК не в свободном доступе для пользователей.

%\item Менеджеры планируют производство исходя из желаемой клиентом даты отгрузки, при этом происходит перегруз производства в отдельные дни. 

%\item У менеджеров нет стандартных вариантов композиции (устаревший вариант композиции) по сырью при создании новых изделий. Менеджер согласовывает композиции сырья по аналогии с другими схожими изделиями, что не всегда экономично.


%\item Менеджеры ориентируются на выпуск продукции только по документам <<Поступление ТМЦ>> в программе СБИС. Кроме того, менеджеры не видят остатки продукции в производстве.

%\item Менеджеры по сути занимаются оперативным планированием производства, указывая последовательность выполнения заказов.

%\item Паспорта качества составляются под каждую позицию, заполняются на складе готовой продукции и не содержат реальной информации по качеству продукции.

%\item Планированием сырья (бумага, картон) занимается генеральный директор. При этом есть в структуре производства выделены отдел Бюро планирования и подготовки производства, Отдел закупок, в которых прописан схожий функционал.
  
\end{enumerate}
% \begin{enumerate}
% %
% %\item Большое количество складских остатков.
% %\item Менеджеры планируют производство исходя из желаемой клиентом даты отгрузки, при этом происходит перегруз производства в отдельные дни. 
% %\item Производство для выполнения графика отгрузки вынуждено производить продукцию заранее, что приводит к увеличению складских остатков по ГП.
% %\item Готовая продукция после выработки не является готовой до момента ее упаковки. Операция упаковки может выполняться значительное время, что может приводить к задержке отгрузки.

% \item Не все ТК подписаны клиентами.

% \item ТК не свободном доступе.

% \item   Не всегда выдаются ТК в производство, зачастую настраивают линию по заявке.


% \item  При внесении изменений, ТК, хранящиеся в папке на линии не актуализируются, риск выпуска продукции оп устаревшей ТК.


% \item  При поступлении оснастка не проверяется.

% \item  Оснастка не пронумерована.

% \item   Списки оснастки отсутствуют.

% \item  Выдача и учет оснастки отсутствует.

% %\item Отсутствует точка учета по списанию (перемещению) материалов в производство, прежде всего бумаги и картона. 
% \item  Не ведется фиксация расследования претензий.

% \item Отсутствие планирования работы цеха № 3.

% \item  Работа в нескольких EXCEL таблицах при планировании гофроагрегата, риск потери данных при переносе.

% %\item Нет четко выстроенных бизнес процессов. Одни и те же функции выполняются разными сотрудниками по-разному. Нет четкого требования и понимания как тот или иной процесс должен происходить.
% \item Ручной ввод данных в 1С «управленку» по планировании гофроагрегата, риск потери данных.

% \item е организован счет Z-картона, считают фистоны вручную, не достоверно.
% %\item Каждый производственный заказ рассматривается как новый, даже если это повторяющиеся изделие. Это приводит к повторному выполнению цепочек бизнес-процессов. На каждый заказ заново формируется печатный пакет документов, включая технологическую карту.

% %\item На производственных линиях отсутствует какой-либо учет производства, журнала простоев и остановов. 
% \item В помещении цеха № 1 при производстве гофрокартона ненадлежащие условия по влажности и температуре.


% \item  Подсчёт заготовок ведет сотрудник ОТК в составе бригады, конфликт интересов.

% \item Цех № 3 не ведут простои и замечания по линиям.
% %Бухгалтерия фиксирует цену и спецификации в 1С.
% %, при этом этой информацией никто не пользуется. 

% \itemЦех № 3 не ведется учет краски и вспомогательных материалов.



% \item Цех № 2 отсутствие на готовой продукции ярлыка, не товарный вид.

% \item Цех № 2 не ведется учет вспомогательных материалов.

% %\item У менеджеров нет стандартных вариантов композиции по сырью при создании новых изделий. Менеджер согласовывает композиции сырья по аналогии с другими схожими изделиями, что не всегда экономично.
% %\item Мастеру гофроагрегата разрешено делать замену сырья на его усмотрение. В момент отсутствия мастера, также может поменять сырье бригадир. %Если замена сырья была произведена, то сначала это на бумаге указывается, а потом в конце смены в 1С вручную меняется в фактическом раскрое.
% %Замена сырья происходит постоянно, если подряд идет в раскроях похожее сырье. Так как плановик сначала делает раскрои, а только потом выстраивается их порядок, то плановик и не может заранее в своем раскрое указать, требуемую замену сырья. В итоге получается, что плановик подкраивает друг с другом заказы с учетом стоимости материалов, а в цеху фактически это игнорируют.
% %\item При поступлении сырья сразу же после оформления документа поступления кладовщик делает корректировку документом 1С <<Корректировка>>, где указывает уже м2 из приходного документа, а вес при этом переводится автоматически по расчетной формул. В итоге в системе 1С: УПП сырье хранится  не в том объеме, как реально поступало.
% %\item Списание сырья на ГА машинистами происходит по метражу, в 1С: УПП машинисты заносят информацию в метраже, а экономисты списывают в тоннах по бумажным раскроям. Как следствие -- некорректные списание и остатки по сырью.
% %\item Менеджеры ориентируются на выпуск продукции только по документам <<Перемещение ТМЦ>> в программе 1С: УПП, при этом не используют типовые отчеты контроля остатков. Кроме того, менеджеры не видят остатки продукции в производстве.
% %\item Менеджеры согласовывают дату производства заказа с отделом планирования по телефону, при этом в 1С: УПП есть функционал, который не используется и позволяющий это автоматизировать. 
% %\item Менеджеры планируют производство исходя из желаемой клиентом даты отгрузки, при этом происходит перегруз производства в отдельные дни. При этом производство для выполнения графика отгрузки вынуждено производить продукцию заранее, что приводит к увеличению складских остатков по ГП.
% %В другие дни производство может быть существенно недозагружено.
% %
% %\item Перемещение на склад делается на основании выпуска, но из-за очереди на паллетайзер образуется временной провал. Получается, что по документам позиция уже есть на остатке, а фактически еще находится в цеху. Из-за этого встречаются проблемы на этапе отгрузки.
% %
% %\item Некорректные данные по остаткам на складам готовой продукции. Учет ведется в 3 местах. 
% %%Факт приемки на склад не ведется.
% %
% \item При выписке документов ТТН на машины, загруженные ночью машинистами, не сверяется количество, вероятность расхождения с документами.
 

% %\item Мастер цеха вручную отражает выпуск в четырех местах (отчет по выпуску, перемещение, журнал выпуска), после этого информация также вручную переносится  еще в три места (два документа Excel и 1С Бухгалтерия и 1С Торговля и Склад). Менеджеры вносят одни и те же данные в коммерческое предложение, договор поставки, спецификацию к договору.
% %По складу материалов (бумага и картон) информация дублируется в 5 (!) местах.
% %
% %\item Паспорта качества составляются под каждую позицию и заполняются вручную и не содержат реальной информации по качеству продукции.
% %. Любое изменение внешнего вида повлечет за собой исправления в 1500 элементах (количество актуальных изделий). Паспорта качества формируются печатаются на складе при отгрузке, при этом информация по партиям, указанная в паспорте, не соответствует действительности.
% %
% %%\item Учет ведется в 3-х местах - рукописные журналы, Excel и 1С УПП. Информация расходится и не понятно где она верна.
% \item Кладовщик цеха № 2 не владеет остатками готовой продукции.
% % (либо есть доступ, но информация не актуальна).
% %
% %\item Нет четко выстроенных бизнес процессов. Одни и те же функции выполняются разными сотрудниками по-разному. Нет четкого требования и понимания как тот или иной процесс должен происходить.
% %
% %\item Контроль производства производится только по звонку или лично. 
% %
% %\item При наличии сложной производственной автоматизированной системы управления (1С: УПП) четко не автоматизирован учет сырья и готовой продукции.
% %
% %\item Велика вероятность недостоверных данных на складских службах.
% %
% %\item На предприятии отсутствует технологическая ''узаконненная'' композиция сырья, в которой должны быть прописаны виды выпускаемого картона и композиции, гарантирующие выпуск продукции указанной марки.
% %\item Отсутствует учет брака.
% %\item Отсутствует учет макулатуры.
% %
% %\end{enumerate}

% %\begin{enumerate}
% %
% %%\item Контрагенты закреплены за разными менеджерами: в отделе маркетинга и в отделе сбыта и логистики, что затрудняет работу покупателей.
% %%\item Контроль производства производится только по звонку или лично.
% %%\item Планирование закупок выполняется вручную, при этом в системе 1С УПП есть достаточно сильный функционал.
% %%\item Приоритет выполнения производственных заказов не ясен.

% %%
% %%\item При наличии эффективной системы планирования раскроев ПОЯС в цехе №3 раскрои и планирование гофроагрегата производится вручную.
% %%\item При ручном планировании сырья обнаружены достаточно большие объемы складских запасов, часть из которых являются "замороженными" достаточно долго.
% %%
% %
% %
% %%\item Большое количество управленческого персонала. Норма управляемости близка к 1 (43 работника ИТР на 53 работника в производстве).

% %%\item В отделе сбыта очень долгий расчет цены, основанной на себестоимости. Для заказчика нужен предварительный расчет цены.
% %%\item Отсутствует предварительное планирование отгрузок.
% %%\item Планирование сырья выполняется коммерческим директором без учета текущего плана производства, складских запасов, планов поступления материалов и прогноза производственных заказов.
% %%\item Отставание контроля оплаты в отделе сбыта. Контроль оплаты производится на основании банковских выписок.
% %%\item Менеджеры отдела сбыта создают производственный заказ не обладая на то полномочий, не владея производственной ситуацией.
% %%\item Контроль производства производится только по звонку или лично.
% %%\item Планирование закупок выполняется вручную, при этом в системе 1С УПП есть достаточно сильный функционал.
% %%\item Отдел ТКО выполняет несвойственные для него функции по расчету себестоимости изделий и расчету раскроев.
% %%\item Стоимость оснастки и доставки не включается в себестоимость изготавливаемой продукции.
% %\item Отсутствуют руководящий персоналии, ответственные управление  производством (Директор по производству).
% %%\item Каждый производственный заказ рассматривается как новый, даже если это повторяющиеся изделие. Это приводит к повторному выполнению цепочек бизнес-процессов. Более 40\% изделий являются повторяющимися и не нуждаются в повторном расчете.
% %%\item В ряде подразделений формируются производственные отчеты в MS Excel, при этом в системе 1С УПП есть типовые отчетные формы с теми же данными.
% %%\item Не найдено использование артикула для номенклатуры, сложный механизм которого реализован в ПЭО.
% %%\item Менеджеры отдела продаж не формируют и не контролируют продажные цены и при выполнении каждого производственного заказа требуют выполнение расчета цены в ПЭО.
% %%\item Отсутствует долгосрочное производственное планирование.
% %%\item Отсутствует планирование работы производства в разрезе позаказного учета.
% %%\item Отсутствует сводное планирование потребности в сырье (долгосрочное и оперативное).
% %%\item Ручное дублирование в журналах при наличии электронных документов в системе 1С УПП.
% %%\item Приоритет выполнения производственных заказов не ясен.
% %%\item В журналах выработки указывается значение из плана, а не фактическое. Таким образом реальная выработка не известна.
% %%\item Расчет себестоимости продукции основан на недостоверных данных.
% %%\item На производственных линиях отсутствует какой-либо учет производства.
% %%В следствие этого нет учета материалов, использования оснастки. Нет возможности выполнять планово-профилактические работы по пробегу оборудования.
% %%\item Требования по сырью на склад формируются в бумажном виде, хотя есть функционал в системе 1С УПП.
% %%
% %
% %\item Программа <<стоп машина>>, предназначенная для контроля простоев, не работает, так как после остановки или поломки все заняты восстановлением и забывают нажать кнопку на компьютере, который находится в комнате мастера ГА.
% %\item Фиксация ремонта и пробег оснастки нигде не учитываются и не ведётся. Состояние оснастки определяется только по факту.
% %\item При изменении дизайна в существующем заказе нет связи между производством (цехом) и ответственными за поступление оснастки на производство (таковые также не определены). Возможен риск изготовления продукции на старой оснастке.
% %\item Для контроля качества выделена лаборатория, но при этом не ведется контроль качества по партиям. Паспорт качества не привязан к конкретной партии и не содержит показателей качества.
% %\item Готовая продукция на складах хранится по ячейкам, но по номенклатуре. Нет информации по заказам покупателей, что приводит к <<залеживанию>> некоторых паллет.
% %\item План производства разрабатывается для перерабатывающих линий в метрах квадратных, а производительность оборудования определена в штуках.
% %\item Не выявлен контроль расхода краски.
% %\item В системе 1С: УПП и на производстве выполнена попытка учета списания сырья по производственным заказам. На выходе экономист формирует отчет <<Справка о проценте брака по сменам>> (\ref{pic:23_wastereport}). Но по факту списание сырья в 1С: УПП идет <<котловым>> методом на весь выпуск продукции. При этом в отчет фактические данные формируются только по выпуску в м2 и фактический расход сырья. 
% \item Начальник цеха № 2 не владеет остатками заготовки в цеху.

% \item Очень большие и бесконтрольные отходы в цеху № 2.

% \item  Образцы для лаборатории перед испытаниями хранятся ненадлежащим образом, результаты испытаний не правдоподобны.

% \item   Кладовщики не материально ответственные.

% \item При проведении обследования не найдено оценки отходов по производству.

% \end{enumerate}


















\section{Предварительные рекомендации по реорганизации производственных процессов}

%%%\item Выделить процесс разработки технологической карты, 
% по проектированию и разработке конструкции, дизайна, технологии изготовления готовой продукции, согласования технологических карт изделий, нормированию технологических операций изготовления продукции, нормированию материалов.



% %\item Выделить функции кладовщика-учетчика по учету перемещения сырья в производство
% \item Выделить физически или логически склад на ГА для перемещения сырья.

% \item Контроль складских остатков, управление складом (поступление, перемещение, списание в производство) перенести полностью на складскую службу без промежуточных звеньев.

% \item Сократить количество согласований в технологическом отделе.
% \item Убрать дублирование процессов, прежде всего по складскому учету материалов и готовой продукции.
 %\end{itemize}



\subsection{Изменение производственных процессов}

\subsubsection{Общие рекомендации}

\begin{itemize}
    \item Разработать и утвердить штатное расписание, структуру предприятия;
    \item Разработать и утвердить список должностных инструкций сотрудников;
    \item Разработать и утвердить стандартные операционные процедуры по каждому бизнес-процессу и производственному процессу.
    \item Настроить сбор информации по каждой точке учета в бумажном (бумажные журналы) или простейшем электронном виде (общие файлы и таблицы).
    \item Организовать рабочие места по каждому элементу штатного расписания.
\end{itemize}

\subsubsection{Подготовка производства}

 \begin{itemize}
 \item Разработать и утвердить перечень возможных композиций сырья для технологических карт. 

\item Разработать и утвердить регламент разработки новых изделий по требованиям заказчиков, поступающих через менеджеров по продажам. 

\item Разработать нормативы технологических операций, переделов.
 
% %Перейти к планированию раскроев ''по композиции", что увеличит возможности работы гофроагрегата.
 \item Реализовать учет выдачи оснастки на производство.
 
\item Реализовать учет ремонта оснастки.

 %\item Организовать доступ к технологическим картам в электронном виде на производстве и всем заинтересованным лицам.
 \end{itemize}

\subsubsection{Предварительное планирование производства и Управление продажами}
\begin{itemize}
\item Запустить процесс предварительного планирования производства, который позволит равномерно загружать мощности без скачков.
\end{itemize}

% \subsubsection{Управление продажами}
% \begin{itemize}
% \item Перевести функции по созданию новых заказов от покупателей и производственных заказов от отдела учета менеджерам отдела продаж и закупок.
% \end{itemize}

\subsubsection{Оперативное планирование производства}
\begin{itemize}
%\item Объединить функции планирования работы гофроагрегатов и перерабатывающих линий. 
\item Утвердить композиционный состав марок.

%\item Планирование работы производства цеха вменить одному работнику в должности планировщика технологического отдела.
%\item Учитывать при планировании приоритеты изготовления производственных заказов.
%%\item  Изменить схему планирования работы производства на "от обратного": сначала должен быть разработан план работы перерабатывающих линий, затем планируется работа гофроагрегата. Это приведет к нормализации загрузки производства и снижения объемов незавершенного производства в виде заготовок в цехе.
%\item Планирование работы производства цеха вменить одному работнику в должности планировщика.
\end{itemize}

\subsubsection{Планирование материалов}

\begin{itemize}
\item Разработать и утвердить не снижаемый запас ТМЦ.
\end{itemize}


 \subsubsection{Управление и диспетчеризация производства}

 \begin{itemize}
\item Организовать учет сырья на ГА. Установить весы, для весового контроля остатков рулонов сырья, снятых с ГА.
 \item Организовать доступ к данным диспетчеризации всем заинтересованным пользователям для повышения оперативности учета производства.
 \item Повысить оперативность учета производства через оперативный ввод выработки на всех скоростных линиях переработки и гофроагрегате.
% На момент проведения обследования достоверная информация о производстве и складских запасов может быть получена лишь к 10 часам следующего за производством дня.

 \item Организовать оперативный контроль гофроагрегата и линий переработки. 
% %Реализовать автоматическую, а не ручную, фиксацию простоев оборудования.
\item Организовать учет ремонта и пробега оснастки, что позволит повысить эффективность ее использования. 
\item Повысить качество обслуживания технологического оборудования. Разработать процедуры и регламент проведения плановых ремонтов оборудования.
% %\item Поставить второй датчик и перенести счетчик заготовок на гофроагрегате, либо поставить выносной экран/монитор для счетчика заготовок.
 \item Фиксировать брак непосредственно на гофроагрегате и линиях переработки.
% %\item Ввести должность руководителя производства (директор, начальник производства).
% \item Организовать учет использования оснастки для повышения качества ее использования.

% %На момент проведения обследования достоверная информация о производстве и складских запасов может быть получена лишь к 10 часам следующего за производством дня.
 \end{itemize}

 \subsubsection{Контроль качества готовой продукции}

 \begin{itemize}
% \item Убрать со службы качества выполнение несвойственных ей функций: учет брака, распечатка ярлыков на ГП, создание пакета упаковки.
% \item Проведение лабораторных испытаний в виде измерения температуры гелеобразования клея - испытания не корректные, т.к. проводятся не на специализированном оборудовании "сделанные на глаз".

 \item Добавить регистрацию показателей качества готовой продукции для паспортов качества в информационных системах для формирования электронных паспортов качества.
 \item Паспорт качества должен формироваться по данным лаборатории.
\end{itemize}

% \subsubsection{Учет сырья и готовой продукции}
% \begin{itemize}
% %\item Реализовать контроль качества по партиям готовой продукции в лаборатории.
% %
% \item Организовать оперативный учет сырья и готовой продукции на основании приходных и расходных документов средствами информационных систем. Ручное ведение учета требует дополнительного контроля, гарантирует наличие ошибок в учете и требует проведения частой инвентаризации.
% \item Устранить учет сырья и готовой продукции в электронных таблицах MS Excel и ручных бланках.
% %\end{itemize}


% %\begin{itemize}
% %\item Отказаться от корректировки поступления сырья по м2, что приводит к недостоверным данным по остаткам сырья на складах.
% %\item Выделить материально-ответственных лиц на производстве по учету материалов и готовой продукции.
% \item Организовать оперативный учет сырья и готовой продукции на основании приходных и расходных документов средствами информационных систем. 
% %Ручное ведение учета требует дополнительного контроля, гарантирует наличие ошибок в учете и требует проведения частых инвентаризаций.
% %%\item Возможна организация ведения сырья и готовой продукции на складе в разрезе штрих-кодов.
% %%\item Возможна детализация учета списания материалов в производство по суткам, бригадам, оборудованию, заказам. На момент обследования в бухгалтерии сырье списывается целиком за период на все выполненные производственные заказы.
% \end{itemize}

%\subsubsection{Информационные системы и технологии}

%%\item Привести основные НСИ (контрагенты, номенклатура, технологические карты) в единую систему для унификации и устранения дублирования.
%\item Реализовать автоматический обмен информацией по одним и тем же процессам между информационными системами.
% \item Повысить достоверность и оперативность ввода данных в систему.
% \item Отказаться от многочисленного дублирования информации в таблицах Excel.
%\end{itemize}




\ifx \notincludehead\undefined
\normalsize
\end{document}
\fi
