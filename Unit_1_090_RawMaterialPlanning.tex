\newpage
\subsection{Планирование сырья}
\label{bp:RawMaterialPlanning}

\textbf{Планирование бумаги и картона}

Планирование основного сырья (бумага и картон) выполняется главным технологом. 

Нормативы по сырью на момент проведения аудита не разработаны. 

Основное сырье (бумага и картон) поставляется предприятием ООО ''Эколайнер''.


% \textbf{Планирование закупки заготовок}


% % \todo{перечитать с планированием}



\textbf{Планирование вспомогательных материалов}

На момент проведения аудита не регламентировано.

% Планированием и закупкой вспомогательных материалов занимается отдел снабжения.

% Каждый день техник по учету определяет остатки по крахмалу и другим материалам (Пленка, скотч и др.) в производстве и сообщает по телефону в отдел снабжения остатки на складе. 25 числа каждого месяца менеджер по снабжению заказывает поставки крахмала на следующий месяц. Объемы заказа определяются коллективно с техником по учету. Менеджер отдела снабжения обзванивает поставщиков по ценам, формирует заявку в свободной форме. Поставки крахмала выполняются 2-3 раза в месяц.

% В системе СБИС хранятся текущие остатки на складах, но отдел снабжения не использует систему СБИС.


% По закупке других материалов (СИЗ, спецодежда, комплектующие и запчасти) отдел снабжения собирает заявки от подразделений, обрабатывает их, ищет поставщиков и производит закупку.


% Поддоны.

% Начальник склада каждый вечер определяет потребность по поддонам и сообщает по телефону в отдел снабжения, где менеджеры отдела снабжения фиксирует вручную объемы потребности.
% Коммерческий отдел и отдел снабжения совместно определяют потребность в поддонах исходя из плана производства и текущих остатков на складах и заказывают поддоны у производителей.
% Менеджеры отдела снабжения создают заявку на закупку поддонов и спецподдонов. Все поддоны невозвратные.
% Поддоны принимаются на складе. Учет поступления фиксируется в системе СБИС.


\textbf{Планирование краски }

На момент проведения аудита не регламентировано.





\clearpage
\input {enddoc}